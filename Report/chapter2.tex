\chapter{LỰA CHỌN PHƯƠNG ÁN}
\section{Thông số đầu vào}
Thiết kế máy thử nghiệm độ bền tĩnh cho dây đỡ cả người theo tiêu chuẩn TCVN 7802-1:2007 với các yêu cầu sau:
\begin{itemize}
\item Lực tối đa tác dụng lên dây đai: 15 kN
\item Phạm vi làm việc của thiết bị đo lực: từ 1{,}2 kN đến 20 kN
\item Độ chính xác của thiết bị đo lực: $\pm$ 2\%
\item Thời gian đạt lực: 240 giây, thời gian duy trì lực: 180 giây
\end{itemize}
\section{Lựa chọn phương án cơ khí}
\subsection{Lựa chọn cấu trúc máy}

Dựa trên các máy thử nghiệm hiện có trên thị trường, các thiết kế cấu trúc chính để thử nghiệm độ bền tĩnh của dây đai an toàn có thể được phân loại như sau: 

\subsubsection{Máy thử nghiệm độ bền kéo dây đai an toàn XJ8108C} 

Máy này sử dụng động cơ servo để truyền động cho cơ cấu trục vít -- bánh vít, chuyển động được truyền qua dây đai răng đến vít me chính xác. Đầu kẹp di động được bố trí theo cấu hình dầm công xôn và được dẫn hướng bằng ray trượt tuyến tính.

Thiết kế này chủ yếu phù hợp để thử nghiệm độ bền kéo của vật liệu dây đai—đặc biệt là các loại vải được sử dụng trong dây an toàn—nhằm xác định khả năng chịu tải của chúng.
\begin{figure}[H]
    \centering
    \includegraphics[width=0.5\textwidth]{pictures/chapter1/c1_p04_sodonguyenlyXJS108C.png}
    \caption{Sơ đồ nguyên lý máy XJS108C}
    \label{fig:label}
\end{figure}

\subsubsection{Hệ thống thử độ bền kéo của công ty Fall Protection Laboratory}
Mặc dù cấu trúc bên trong của máy này không được mô tả chi tiết, các tài liệu tham khảo trực quan cho thấy thiết kế. Sự khác biệt chính có thể nằm ở cơ cấu kẹp chặt và kích thước tổng thể của máy.
\begin{figure}[H]
    \centering
    \includegraphics[width=0.5\textwidth]{pictures/chapter2/c2_p01_FPL.png}
    \caption{Sơ đồ nguyên lý máy của công ty FPL}
    \label{fig:label}
\end{figure}
Máy của Fall Protection LAB có kích thước lớn hơn và được trang bị nhiều loại kẹp chuyên dụng được thiết kế để thử nghiệm các thành phần riêng lẻ của hệ thống dây an toàn toàn thân, chẳng hạn như dây đai vải, khóa an toàn và dây giảm chấn.

Bất chấp những khác biệt này, cả hai máy đều có chung một hạn chế: không có máy nào hiện tại có khả năng thử nghiệm toàn bộ hệ thống dây an toàn đỡ người như một tổng thể trong điều kiện làm việc thực tế. Điều này đặt ra một thách thức đáng kể trong việc đánh giá hiệu suất tổng thể và độ an toàn của các hệ thống dây an toàn tích hợp.
\subsubsection{Máy thử dây đai an toàn điện tử KASON ETM504D}
\begin{figure}[H]
    \centering
    \includegraphics[width=0.5\textwidth]{pictures/chapter2/c2_p02.png}
    \caption{Sơ đồ nguyên lý máy KASON ETM504D}
    \label{fig:label}
\end{figure}
\subsubsection{Yêu cầu theo tiêu chuẩn TCVN đối với nguyên lý máy}
Theo tiêu chuẩn TCVN 7802-1:2007, mục 5 "Phương pháp thử", mục nhỏ 5.1.5 "Thiết bị thử độ bền tĩnh" quy định các thành phần cần thiết cho thiết bị thử nghiệm độ bền tĩnh, bao gồm:
\begin{itemize}
\item Khung thử nghiệm
\item Tời hoặc cơ cấu thủy lực
\item Đồng hồ đo
\item Thanh ngang phù hợp để tạo được tải lên mẫu thử mô phỏng theo nửa thân người
\end{itemize}
Do đó, thiết kế cơ khí của máy phải tuân theo tiêu chuẩn TCVN và sử dụng cơ cấu tời làm cấu trúc chính của máy, điều này có hiệu quả về chi phí. Hơn nữa, trong máy này không cần có độ chính xác vị trí cao, chỉ cần độ chính xác cho lực và thời gian để đạt được lực đó. Với những yêu cầu này, chúng ta có thể giải quyết bằng cách sử dụng các thuật toán điều khiển như PID để đảm bảo điều khiển lực.
Sơ đồ nguyên lý mà nhóm lựa chọn:
\begin{figure}[H]
    \centering
    \includegraphics[width=0.7\textwidth]{pictures/chapter2/c2_p03_sodonguyenly.png}
    \caption{Sơ đồ nguyên lý phương án lựa chọn}
    \label{fig:label}
\end{figure}

\subsection{Lựa chọn tời kéo}

Dựa vào trang web của công ty EMCÉ winches, các loại tời kéo có sẵn trên thị trường:
\begin{itemize}
\item Tời kéo điện
\item Tời kéo thủy lực
\item Tời kéo khí nén
\end{itemize}

Mỗi loại có các ưu nhược điểm khác nhau:
\begin{table}[H]
\centering
\caption{So sánh các loại tời}
\begin{tabular}{|p{2cm}|p{6cm}|p{6cm}|}
\hline
\textbf{Loại tời} & \textbf{Ưu điểm} & \textbf{Nhược điểm} \\
\hline
Tời điện & 
\begin{itemize}[leftmargin=*,nosep]
\item Dễ kết nối với lưới điện dân dụng
\item Chống nước và bụi (lên đến IP67)
\item Có thể tích hợp hệ thống làm mát và cảm biến
\item Phù hợp cho các ứng dụng khác nhau
\item Dải công suất rộng
\end{itemize} &
\begin{itemize}[leftmargin=*,nosep]
\item Có thể cần thêm biến tần hoặc phanh cho một số ứng dụng
\item Hiệu suất thấp trong giảm tốc
\item Khả năng chịu tải thấp hơn so với tời thủy lực
\end{itemize} \\
\hline
Tời thủy lực &
\begin{itemize}[leftmargin=*,nosep]
\item Được ưu tiên khi có sẵn hệ thống thủy lực
\item Kích thước nhỏ gọn so với công suất đầu ra
\item Chống nước và bụi
\item Bền và đáng tin cậy
\item Thường được sử dụng cho các ứng dụng công suất/mô-men cao như hàng hải
\end{itemize} &
\begin{itemize}[leftmargin=*,nosep]
\item Thường ồn ào
\item Yêu cầu hệ thống thủy lực phức tạp
\item Chi phí lắp đặt cao hơn
\end{itemize} \\
\hline
Tời khí nén &
\begin{itemize}[leftmargin=*,nosep]
\item An toàn trong môi trường dễ nổ (dầu, khai thác mỏ, hóa chất)
\item Khả năng chịu quá tải tốt
\end{itemize} &
\begin{itemize}[leftmargin=*,nosep]
\item Điều khiển kém linh hoạt so với tời điện
\item Yêu cầu nguồn cung cấp khí nén
\end{itemize} \\
\hline
\end{tabular}
\end{table}

\textbf{Yêu cầu đối với tời:}

\begin{itemize}
\item Phải cung cấp lực kéo 15kN thông qua hệ thống truyền động.
\item Thời gian đạt lực: 240 giây, thời gian duy trì lực: 180 giây.
\end{itemize}

Đối với thiết kế máy thử nghiệm độ bền tĩnh, tời điện là lựa chọn tối ưu do khả năng đáp ứng yêu cầu lực kéo, điều khiển linh hoạt và phù hợp với điều kiện phòng thí nghiệm. Tời điện có thể tạo ra lực kéo ổn định, cho phép điều khiển tốc độ để đạt được lực yêu cầu trong vòng 240 giây theo tiêu chuẩn thử nghiệm.

So với hệ thống thủy lực hoặc khí nén, tời điện có ưu thế rõ ràng vì không yêu cầu thay đổi cấu trúc hoặc nguồn cung cấp phụ trợ bổ sung. Điều này tối ưu hóa chi phí và đảm bảo vận hành ổn định trong môi trường phòng thí nghiệm với lưới điện tiêu chuẩn. Vì vậy lựa chọn tời điện là phù hợp nhất cho ứng dụng này.

\subsection{Lựa chọn thành phần kết nối}

Theo TCVN 7802-1:2007, Mục 5.1.4 "Giá thử", điểm gắn kết trên khung thử nghiệm được quy định như sau:

\begin{itemize}
\item Điểm móc dây phải là một vòng tròn có đường kính lỗ là $(20 \pm 1)$ mm và đường kính mặt cắt ngang là $(15 \pm 1)$ mm, hoặc một thanh truyền có đường kính mặt cắt ngang tương tự.
\end{itemize}

Trong mục 5.1.1 và 5.1.2 "Mẫu thử mô phỏng theo nửa thân người", được quy định:

\begin{itemize}
\item Mẫu thử mô phỏng theo nửa thân người như hình bên dưới, đinh khuy treo phải đường kính trong 40 mm, được sử dụng để kết nối mẫu thử với khung thử nghiệm hoặc dây treo trong quá trình thử nghiệm.
\end{itemize}

Trong thực tế, các nhà sản xuất dây đỡ cả người thường thiết kế các phần tử gắn kết chống rơi dưới dạng vòng thép để kết nối với dây treo. Điều này làm nổi bật tầm quan trọng của việc lựa chọn các thành phần kết nối tuân thủ tiêu chuẩn thử nghiệm.
\begin{figure}[H]
    \centering
    \includegraphics[width=0.5\textwidth]{pictures/chapter2/c2_p04.png}
    \caption{Minh họa đinh khuy treo của dây đỡ cả người}
    \label{fig:label}
\end{figure}

Do đó, để thiết kế máy thử nghiệm độ bền tĩnh, móc treo là lựa chọn tối ưu. Móc treo không chỉ đáp ứng các yêu cầu kỹ thuật về kích thước và kết nối mà còn đảm bảo độ an toàn cao trong quá trình thử nghiệm cả mẫu thử mô phỏng theo nửa thân người và mẫu dây đỡ cả người. Ngoài ra, việc sử dụng móc treo giúp đơn giản hóa quá trình lắp đặt và tăng tính linh hoạt trong quá trình thử nghiệm.

\cleardoublepage
\section{Lựa chọn phương án điện}
Máy thử nghiệm đồ bền tĩnh DĐCN bao gồm bộ điều khiển và phản hồi bằng cảm biến loadcell kèm bộ khuếch đại. Sau đó, bộ điều khiển chính đảm bảo lực đầu ra bằng cách gửi tín hiệu đến tời điện để điều khiển lực. Máy sử dụng cảm biến tiệm cận để dừng khi vượt ra ngoài vùng làm việc.

Nhìn chung, sơ đồ điện của máy thử nghiệm độ bền tĩnh cho dây đỡ cả người được trình bày dưới đây:
\begin{figure}[H]
    \centering
    \includegraphics[width=0.5\textwidth]{pictures/chapter2/c2_p05.png}
    \caption{Sơ đồ điện}
    \label{fig:label}
\end{figure}
\subsection{Lựa chọn cảm biến lực}
Để đọc giá trị lực trong quá trình kéo, loadcell được sử dụng. Việc lựa chọn loại loadcell phù hợp cho lực kéo và độ chính xác có thể đảm bảo thiết kế đáp ứng yêu cầu sử dụng trong khi vẫn tiết kiệm chi phí. Các lựa chọn loadcell sau được xem xét:

\begin{itemize}
\item Loadcell đơn điểm: Loại loadcell phổ biến nhất, có khả năng chịu nén ở tải trọng thấp. Cung cấp độ chính xác và độ tin cậy tốt, đo tải trọng không tập trung với một đầu cố định và một đầu chịu tải.

\item Loadcell dầm: Bao gồm ba loại chính: dầm uốn, dầm cắt và dầm cắt hai đầu.
\begin{itemize}
\item Loadcell dầm uốn: Tiết kiệm chi phí và linh hoạt, phù hợp cho các ứng dụng cân trọng lượng tải nhỏ như bể nước hoặc container.
\item Loadcell dầm cắt: Phù hợp cho phạm vi tải từ thấp đến trung bình, được sử dụng trong các ứng dụng như bể nước hoặc container.
\item Loadcell dầm cắt hai đầu: Cố định ở cả hai đầu với lực tác dụng ở giữa, phù hợp cho tải trọng cao hơn, được sử dụng trong cân bể hoặc cân xe tải.
\end{itemize}

\item Loadcell bánh xe (pancake): Thiết kế mỏng, hình tròn phù hợp cho đo lực từ trung bình đến cao. Có thể được lắp đặt giữa hai bộ phận để đo nén hoặc kéo thông qua các lỗ ren, với phép đo chính xác ngay cả khi có tải trọng lệch trục.

\item Loadcell chốt: Được sử dụng thay thế cho chốt hoặc trục chịu lực, có khả năng chịu lực rất cao.

\item Loadcell ống: Một trong những thiết kế loadcell sớm nhất, có khả năng đo kéo hoặc nén, đặc biệt phù hợp cho các ứng dụng lực cao như cân xe tải hoặc cân đường sắt.

\item Loadcell chữ S: Phù hợp cho tải trọng từ thấp đến trung bình, đo cả kéo và nén, thường được sử dụng trong các ứng dụng treo. Thiết kế với hai lỗ ren để kết nối giữa các thành phần.

\item Loadcell liên kết kéo: Rất linh hoạt cho tải trọng từ thấp đến cao, thường được sử dụng trong cân cẩu, hệ thống tời và thử nghiệm độ bền kéo.
\end{itemize}

Do đó, các loại loadcell phù hợp cho ứng dụng lực kéo này bao gồm: loadcell bánh xe, loadcell chữ S và loadcell liên kết kéo.

\textbf{Yêu cầu:}
\begin{itemize}
\item Lực kéo tối đa trên dây đai: 15 kN
\item Độ chính xác của cảm biến lực: $\pm$ 2\%
\item Phạm vi làm việc của thiết bị đo lực: từ 1{,}2 kN đến 20 kN
\item Dễ dàng lắp đặt và thay thế
\item Phù hợp cho thử nghiệm thường xuyên
\end{itemize}

\textbf{Lựa chọn:} Loadcell chữ S được chọn vì những lý do sau:
\begin{itemize}
\item Đáp ứng phạm vi lực và độ chính xác yêu cầu.
\item Được sử dụng rộng rãi trong máy thử nghiệm lực kéo.
\item Chi phí thấp, sẵn có, dễ sản xuất và bảo trì.
\item Dễ dàng lắp đặt với khung máy.
\end{itemize}

\subsection{Lựa chọn cảm biến an toàn}
Để giới hạn khoảng cách di chuyển an toàn cho các kẹp và ngăn ngừa lỗi vận hành, đảm bảo an toàn cho người dùng và cơ khí, các lựa chọn cảm biến phổ biến bao gồm công tắc hành trình, cảm biến tiệm cận, laser và cảm biến hồng ngoại.
\begin{table}[H]
\centering
\caption{So sánh các loại cảm biến}
\begin{tabular}{|p{3.5cm}|p{5.5cm}|p{5.5cm}|}
\hline
\textbf{Loại cảm biến} & \textbf{Ưu điểm} & \textbf{Nhược điểm} \\
\hline
Công tắc hành trình & 
\begin{itemize}[leftmargin=*,nosep]
\item Dễ dàng lắp đặt và định vị
\item Dễ đọc tín hiệu
\item Chi phí thấp
\end{itemize} &
Tuổi thọ thấp do tiếp xúc cơ khí \\
\hline
Cảm biến tiệm cận &
\begin{itemize}[leftmargin=*,nosep]
\item Độ bền cao
\item Dễ dàng lắp đặt và định vị
\item Dễ đọc tín hiệu
\end{itemize} &
\begin{itemize}[leftmargin=*,nosep]
\item Chi phí trung bình
\item Dễ bị nhiễu trong môi trường nhiệt độ cao
\end{itemize} \\
\hline
Cảm biến Laser &
\begin{itemize}[leftmargin=*,nosep]
\item Độ bền cao
\item Dễ đọc tín hiệu
\end{itemize} &
\begin{itemize}[leftmargin=*,nosep]
\item Lắp đặt phức tạp so với công tắc hành trình và cảm biến tiệm cận
\item Chi phí cao
\end{itemize} \\
\hline
Cảm biến hồng ngoại &
\begin{itemize}[leftmargin=*,nosep]
\item Độ bền cao
\item Dễ đọc tín hiệu
\end{itemize} &
Lắp đặt phức tạp \\
\hline
\end{tabular}
\end{table}
\textbf{Kết luận:} Lựa chọn phương án 2 (cảm biến tiệm cận) do yêu cầu độ bền cao là yếu tố quan trọng. Phương án này cũng đáp ứng các yêu cầu khác như dễ lắp đặt, dễ đọc tín hiệu và chi phí tương đối thấp.

\section{Lựa chọn phương án điều khiển}
\subsection{Lựa chọn phần cứng điều khiển}
Việc lựa chọn bộ điều khiển trung tâm phù hợp là rất quan trọng để đảm bảo máy thử nghiệm hoạt động chính xác, ổn định và tin cậy. Bộ điều khiển chịu trách nhiệm xử lý tín hiệu từ các cảm biến như lực và vị trí, thực thi các thuật toán điều khiển như PID để điều khiển lực kéo, và giao tiếp với các bộ truyền động (biến tần điều khiển động cơ tời) cũng như giao diện người--máy (HMI). Do máy này sẽ được sử dụng thường xuyên trong môi trường thử nghiệm và cần độ chính xác cao, các tiêu chí chính để lựa chọn bộ điều khiển bao gồm: khả năng xử lý tín hiệu cảm biến và điều khiển động cơ, thời gian đáp ứng nhanh, dễ lập trình, độ bền cho sử dụng công nghiệp thường xuyên, khả năng chống nhiễu tốt và chi phí hợp lý. Các lựa chọn chính được xem xét bao gồm vi điều khiển (MCU), hệ thống điều khiển dựa trên PC và bộ điều khiển logic khả trình (PLC).

Sử dụng vi điều khiển (như Arduino, ESP32 hoặc STM32) mang lại lợi thế về chi phí phần cứng thấp và lập trình linh hoạt. Tuy nhiên, đối với máy thử nghiệm này, vi điều khiển có một số hạn chế. Thứ nhất, chúng thường ít mạnh mẽ hơn trong môi trường công nghiệp, nơi nhiễu điện, rung động và biến đổi nhiệt độ là phổ biến—khiến chúng kém tin cậy hơn cho loại ứng dụng này. Thứ hai, việc tích hợp MCU với các thiết bị công nghiệp tiêu chuẩn (như cảm biến 24V, tín hiệu loadcell được khuếch đại và biến tần) thường yêu cầu mạch tùy chỉnh bổ sung, làm tăng độ phức tạp và các vấn đề tiềm ẩn về độ tin cậy. Cuối cùng, mặc dù MCU cung cấp lập trình linh hoạt, việc phát triển các ứng dụng điều khiển công nghiệp phức tạp liên quan đến đa nhiệm và giao tiếp với nhiều thiết bị có thể tốn thời gian và đòi hỏi kiến thức lập trình nhúng sâu—không giống như các ngôn ngữ có cấu trúc và quen thuộc hơn được sử dụng trong lập trình PLC cho các kỹ sư tự động hóa.

Điều khiển dựa trên PC sử dụng máy tính cá nhân kết hợp với thẻ thu thập dữ liệu (DAQ) cung cấp khả năng xử lý mạnh mẽ, giao diện đồ họa phong phú và khả năng lưu trữ và phân tích dữ liệu tuyệt vời. Tuy nhiên, PC tiêu chuẩn không được thiết kế cho các nhiệm vụ điều khiển liên tục, thời gian thực với độ ổn định cao—khiến chúng kém phù hợp cho loại máy thử nghiệm này. Các vấn đề như sự cố hệ thống, lỗi hệ điều hành hoặc thậm chí độ trễ khởi động có thể ảnh hưởng đến hoạt động. Ngoài ra, PC thường nhạy cảm với môi trường công nghiệp (bụi, độ ẩm, nhiễu điện từ, rung động). Mặc dù sử dụng PC công nghiệp (IPC) là một lựa chọn, chi phí hệ thống tổng thể—bao gồm IPC, thẻ DAQ cao cấp và phần mềm chuyên dụng—có thể vượt quá ngân sách. Việc đảm bảo hiệu suất thời gian thực cho vòng lặp điều khiển cũng phức tạp hơn so với hệ thống dựa trên PLC.

Xem xét các yêu cầu và hạn chế của các lựa chọn khác, PLC (Programmable Logic Controller) nổi bật là lựa chọn phù hợp nhất cho máy thử nghiệm này. PLC được xây dựng đặc biệt cho môi trường công nghiệp, cung cấp độ bền cơ khí cao, khả năng chống nhiễu điện từ mạnh và hiệu suất ổn định trong phạm vi nhiệt độ rộng. Các mô-đun I/O của chúng được tiêu chuẩn hóa và thường bao gồm bảo vệ và cách ly tích hợp, cho phép kết nối trực tiếp và đáng tin cậy với các thiết bị công nghiệp. Các PLC hiện đại như Mitsubishi FX3U có tốc độ xử lý đủ để thực hiện điều khiển PID và xử lý tín hiệu cần thiết cho ứng dụng này. Lập trình bằng các ngôn ngữ như Ladder Logic hoặc Function Block Diagram giúp đơn giản hóa việc phát triển cả logic điều khiển tuần tự và logic điều khiển quá trình. Quan trọng nhất, PLC cung cấp giải pháp cân bằng—độ tin cậy công nghiệp đã được chứng minh, hiệu suất tốt và chi phí hợp lý—khiến chúng trở nên lý tưởng cho các nhiệm vụ tự động hóa quy mô nhỏ đến trung bình như máy thử nghiệm này. Vì những lý do này, PLC được chọn làm bộ điều khiển chính. \\

\textbf{Yêu cầu:}
\begin{itemize}
\item Có khả năng xử lý các nhiệm vụ đọc tín hiệu cảm biến và điều khiển biến tần điều khiển động cơ.
\item Tốc độ đọc tín hiệu đáp ứng nhu cầu tính toán.
\item Lập trình đơn giản.
\item Độ bền cao phù hợp cho thử nghiệm công nghiệp thường xuyên.
\item Khả năng chống nhiễu môi trường tốt.
\item Chi phí hợp lý.
\end{itemize}

Một yếu tố quan trọng khác trong việc lựa chọn bộ điều khiển là khả năng đọc và xử lý chính xác tín hiệu loadcell dung lượng cao. Loadcell được sử dụng trong máy thử nghiệm này có định mức lên đến 2 tấn, thường tạo ra điện áp tương tự đầu ra rất nhỏ (ví dụ: 0--10 mV hoặc 4--20 mA qua bộ khuếch đại). Việc đạt được độ phân giải cao và thu thập tín hiệu không nhiễu trong trường hợp này đòi hỏi không chỉ khả năng chuyển đổi tương tự--số (ADC) tốt mà còn cả thiết kế điện ổn định và cách ly tín hiệu thích hợp.

Mặc dù vi điều khiển có thể giao tiếp với tín hiệu loadcell được khuếch đại, hầu hết các bo mạch MCU có sẵn cung cấp độ phân giải ADC hạn chế (thường là 8 đến 12 bit) và thiếu lọc nhiễu cấp công nghiệp tích hợp và cách ly. Điều này khiến chúng dễ bị trôi đo lường và không chính xác hơn trong môi trường nhiễu, đặc biệt là khi yêu cầu đo lực chính xác cao.

Ngược lại, PLC với các mô-đun đầu vào tương tự (ví dụ: độ phân giải 12 đến 16 bit) được xây dựng có mục đích để đọc các tín hiệu như vậy trong các thiết lập công nghiệp thực tế. Các mô-đun này thường bao gồm các tính năng như:

\begin{itemize}
\item Cách ly galvanic tích hợp để ngăn chặn vòng lặp đất,
\item Lọc phần cứng để loại bỏ nhiễu điện,
\item Phạm vi đầu vào có thể cấu hình phù hợp với các cảm biến công nghiệp điển hình.
\end{itemize}

Do đó, đối với các ứng dụng như thế này trong đó đo lực chính xác (lên đến 2 tấn) là yếu tố thiết yếu cho an toàn hệ thống và độ chính xác điều khiển, PLC cung cấp giải pháp mạnh mẽ và đáng tin cậy hơn so với hầu hết các thiết lập dựa trên vi điều khiển.

\textbf{Kết luận:} Lựa chọn PLC được chọn do các ưu điểm phù hợp với các tiêu chí đã đưa ra.

\subsection{Lựa chọn phương pháp điều khiển}
Dựa trên các yêu cầu thử nghiệm được nêu trong tiêu chuẩn TCVN 7802-1:2007, trong đó quy định:

"Thời gian đạt lực: 240 giây, thời gian duy trì lực: 180 giây", chúng ta có thể rút ra biểu đồ lực theo thời gian mà máy thử nghiệm phải tuân theo, như được thể hiện trong hình dưới đây.
\begin{figure}[H]
    \centering
    \includegraphics[width=0.8\textwidth]{pictures/chapter2/c2_p06.png}
    \caption{Motion profile}
    \label{fig:label}
\end{figure}
Để đáp ứng yêu cầu chính của máy thử nghiệm—điều khiển chính xác lực kéo tác dụng lên dây an toàn theo biểu đồ thời gian trong TCVN 7802-1:2007 (tăng lực lên 15 kN trong 240 giây, sau đó giữ ổn định trong 180 giây)—chúng ta cần một phương pháp điều khiển tốt và đáng tin cậy.

Hệ thống sử dụng loadcell để đo liên tục lực, và động cơ tời được điều khiển bởi biến tần (VFD - Variable Frequency Drive), cho phép điều chỉnh tốc độ và mô-men xoắn một cách mượt mà. Tuy nhiên, do các yếu tố như ma sát, quán tính và một số hành vi phi tuyến trong hệ thống, cần có phương pháp điều khiển thông minh để giảm thiểu sai số và giữ lực chính xác. Vì vậy, trong ứng dụng này, chúng ta sử dụng điều khiển PID vì nó có thể đáp ứng tốt các yêu cầu của chúng ta, hơn nữa nó được sử dụng rộng rãi trong tự động hóa công nghiệp vì hiệu quả, đáng tin cậy và tương đối đơn giản để thiết lập. Một lý do quan trọng khác là bộ điều khiển chúng ta sử dụng ở đây là PLC, ngày nay hầu hết PLC đều có khối chức năng PID tích hợp, giúp việc thiết lập và điều chỉnh thuận tiện hơn. Ngoài ra còn có các phương pháp điều chỉnh tham số PID qua thử nghiệm và sai số đã được biết đến rộng rãi.

\subsection{Lưu đồ giải thuật}
Dựa trên quy trình vận hành được nêu trong thiết kế máy, thuật toán chương trình chính toàn diện cho quá trình nâng tải và điều khiển lực được minh họa trong sơ đồ dưới đây. Thuật toán này cung cấp một biểu diễn rõ ràng và có cấu trúc của toàn bộ quy trình, bắt đầu từ thiết lập ban đầu đã thảo luận trước đó. Sơ đồ đóng vai trò như một hướng dẫn trực quan cho các bước quan trọng liên quan đến chương trình xử lý tải, đảm bảo rằng mỗi giai đoạn của quá trình được thực hiện một cách có hệ thống và nhất quán.
\begin{figure}[H]
    \centering
    \includegraphics[width=0.75\textwidth]{pictures/chapter2/c2_p07_luudogiaithuat.png}
    \caption{Lưu đồ giải thuật}
    \label{fig:label}
\end{figure}

Thuật toán bắt đầu bằng việc hạ cần cẩu để định vị móc để gắn mô hình nhân thể 100 kg, một bước quan trọng để bắt đầu hoạt động nâng một cách an toàn. Sau đó, quá trình kiểm tra việc phát hiện cảm biến dưới bằng cảm biến tiệm cận để xác nhận rằng cần cẩu nằm trong vùng làm việc được chỉ định. Nếu cảm biến được kích hoạt, cần cẩu dừng hạ xuống, và móc, hiện đã gắn với mô hình nhân thể 100 kg, được cân bằng với loadcell. Trong quá trình cân bằng này, loadcell ghi lại trọng lượng của mô hình nhân thể làm giá trị cơ sở, đảm bảo đo lực chính xác cho các hoạt động nâng tiếp theo. Sau đó, thuật toán tiến hành bắt đầu nâng mô hình nhân thể, liên tục giám sát lực thông qua cảm biến loadcell.

Trong giai đoạn nâng, thuật toán đánh giá xem giá trị đọc từ loadcell có vượt quá giá trị đã được cân bằng 100kg hay không, cho thấy các lực bổ sung hoặc điều kiện quá tải vượt quá trọng lượng của mô hình nhân thể. Nếu ngưỡng này bị vượt quá, hệ thống bắt đầu một quy trình để tăng lực nâng để quản lý tải một cách hiệu quả. Đồng thời, máy kiểm tra các điều kiện tải khi lực đạt 1500kg, như được điều khiển bởi bộ điều khiển chính và tời điện. Sau đó, thuật toán kiểm tra điều kiện dừng bằng cảm biến tiệm cận để đảm bảo cần cẩu vẫn nằm trong vùng làm việc an toàn. Nếu điều kiện dừng được thỏa mãn—nghĩa là cần cẩu đã di chuyển ra ngoài vùng làm việc—cần cẩu hạ xuống cho đến khi chạm đến cảm biến giới hạn, đảm bảo hoạt động an toàn. Phương pháp có cấu trúc này đảm bảo rằng quá trình nâng tải và điều khiển lực được tiến hành phù hợp với các yêu cầu vận hành của máy, cung cấp hiệu suất đáng tin cậy và an toàn đáp ứng các tiêu chuẩn quy định.

Hơn nữa, các chiến lược điều khiển đơn giản hơn, chẳng hạn như điều khiển vòng hở hoặc điều khiển Bật-Tắt, vốn không thể đáp ứng các yêu cầu về điều chỉnh lực chính xác và ổn định vì chúng thiếu các cơ chế phản hồi cần thiết để giám sát lỗi liên tục và điều chỉnh khắc phục.