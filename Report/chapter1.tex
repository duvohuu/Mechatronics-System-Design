\chapter{TỔNG QUAN}
    
\section{Lý do chọn đề tài và các khái niệm chính}
\subsection{Lý do chọn đề tài}
Trong các môi trường làm việc có rủi ro cao, nơi an toàn tuyệt đối không thể được đảm bảo, hệ thống chống rơi ngã cá nhân (Personal fall-arrest system - PFAS) đóng vai trò quan trọng trong việc bảo vệ người lao động khỏi ngã. Tuy nhiên, thiết bị thử nghiệm lỗi thời và quy trình không nhất quán có thể dẫn đến đánh giá độ bền không chính xác của dây đai an toàn, làm gia tăng rủi ro tại nơi làm việc.

Vì vậy nhóm quyết định chọn đề tài, "Thiết kế máy thử nghiệm độ bền tĩnh cho dây đai an toàn," nhằm phát triển một hệ thống tiên tiến có khả năng đo lường chính xác độ bền tĩnh của dây đai an toàn để đảm bảo chúng đáp ứng các tiêu chuẩn chất lượng nghiêm ngặt. Công nghệ thử nghiệm hiện đại này không chỉ nâng cao an toàn và giảm chi phí liên quan đến tai nạn mà còn củng cố uy tín của nhà sản xuất và hỗ trợ sự phát triển của ngành thông qua việc cải thiện kiểm soát chất lượng và tiêu chuẩn hóa thử nghiệm.
\subsection{Các khái niệm chính}
Một số khái niệm được sử dụng để thử nghiệm theo TCVN 7802-1:2007:
\subsubsection{Các khái niệm liên quan đến dây đeo cả người}
\begin{figure}[H]
    \centering
    \includegraphics[width=1\textwidth]{pictures/chapter1/c1_p01.png}
    \caption{Ví dụ của dây đeo cả người}
    \label{fig:label}
\end{figure}
\begin{itemize}
    \item \textbf{Dây đỡ cả người (DĐCN):} Bộ phận của thiết bị đỡ cả người để giữ người ở trong hệ thống chống rơi ngã cá nhân.
    
    \item \textbf{Dây chính:} Dây thuộc dây đỡ cả người được chế tạo để truyền tải, đỡ cơ thể người hoặc làm giảm áp lực lên người trong quá trình rơi và sau khi sự rơi kết thúc.
    
    \item \textbf{Dây phụ:} Dây được thiết kế cùng với DĐCN, không phải dây chính.
    
    \item \textbf{Chi tiết liên kết chống rơi ngã:} Phụ kiện bắt buộc, được thiết kế như điểm liên kết để kết nối với hệ thống chống rơi ngã.
    
    \item \textbf{Khóa nhanh:} Phụ kiện gồm hai phần được thiết kế để dễ dàng đeo và tháo DĐCN.
\end{itemize}
\cleardoublepage
\subsubsection{Các khái niệm liên quan đến hệ thống}
\textbf{Hệ thống chống rơi ngã cá nhân (personal fall-arrest system) - HTCRN: } Hệ thống được thiết kế để chống rơi ngã từ trên cao, giảm thiểu xung lực khi rơi, kiểm soát toàn bộ khoảng cách rơi để ngăn ngừa việc va vào nền đất hoặc vật cản khác và để giữ người rơi xuống trong một tư thế thích hợp.
\subsubsection{Một số yêu cầu đối với dây đai toàn thân}

\begin{itemize}
    \item Tất cả DĐCN phải tối thiểu là loại A đối với mục đích chống rơi ngã.
    
    \item DĐCN loại A được thiết kế để đỡ cơ thể người trong và sau khi sự rơi kết thúc. Chúng phải có ít nhất một chi tiết liên kết chống rơi ngã. Chi tiết liên kết chống rơi ngã phải được bố trí sao cho nó đặt ở phía sau của người đeo và chính giữa hai dây quàng vai trên, hoặc ở giữa phần trước ngực không gần nhiều cao của xương ức.
    
    \item Tất cả các chi tiết liên kết làm móc treo của vật liệu dệt phải được bảo vệ thích hợp để chống mài mòn, cả ở trong và ngoài móc.
    
    \item Tuỳ thuộc vào sự phân loại DĐCN, các chi tiết liên kết phải được gắn vào DĐCN ở những vị trí được quy định ở 4.2. Theo tiêu chuẩn này, những điểm liên kết ở phía trước và phía bên cạnh để kết nối với hệ thống tai vị trí làm việc không được chấp nhận để sử dụng chống rơi ngã.
    
    \item Chi tiết liên kết chống rơi ngã của DĐCN loại A, khi đặt ở lưng của người đeo và chính giữa hai vai, phải được thiết kế sao cho không trượt xuống lưng của mẫu thử mô phỏng theo nửa thân người trong khi thử động quay ngược.
\end{itemize}

\subsubsection{Yêu cầu về độ bền tĩnh để đảm bảo an toàn cho người sử dụng}

\begin{itemize}
    \item Dây đai toàn thân phải chịu được lực 15 kN trong quá trình thử nghiệm.
    \item Không xé rách vật liệu vải làm đai.
    \item Không làm đứt đường may ở bất kỳ điểm nào.
    \item Không gãy một phần hoặc toàn bộ của bất kỳ khóa nào.
    \item Không mở ngoài ý muốn của bất kỳ khóa mở nhanh nào.
    \item Các dây và các chi tiết liên kết không được di chuyển lệch khỏi vị trí. Các dây được phép trượt qua khóa điều chỉnh, những không quá 25mm.
\end{itemize}
\section{Phương pháp thử nghiệm TCVN 7802-1:2007}
Theo Mục 5 của TCVN 7802-1:2007 về Phương pháp Thử nghiệm, trong Mục 5.1 "Thiết bị" bao gồm: 5.1.1 "Mẫu thử mô phỏng theo nửa thân người để thử tĩnh", 5.1.4 "Giá thử", 5.1.5 "Thiết bị thử độ bền tĩnh", 5.1.8 "Dụng cụ đo lực". Trong Mục 5.4 "Thử độ bền tĩnh đối với các chi tiết liên kết chống rơi ngã" có cung cấp quy định về quy trình thử nghiệm và kết quả các kết quả cần ghi nhận ghi nhận trong quá trình thử nghiệm độ bền tĩnh.
\subsubsection{Quy định về mẫu thử:}
Hình nộm nửa người mô phỏng để thử nghiệm tĩnh phải phù hợp với kích thước được thể hiện trong hình bên dưới. Đinh khuy treo phải có đường kính trong là 40 mm và đường kính mặt cắt ngang tối đa là 16 mm. Bề mặt làm nhẵn, và nếu là kết cấu bằng gỗ, bề mặt phải được đánh bóng bằng senlắc hoặc vécni.
\begin{figure}[H]
    \centering
    \includegraphics[width=1\textwidth]{pictures/chapter1/c1_p02.png}
    \caption{Hình nộm mô phỏng nửa thân người để thử tĩnh}
    \label{fig:label}
\end{figure}

\textbf{Một số quy định về giá thử:}

\begin{itemize}
    \item Có kết cấu neo cứng vững, sao cho tần số rung tự nhiên theo trục thẳng đứng tại điểm móc dây không vượt quá 100 Hz và sao cho khi tác dụng một lực 20 kN trên điểm móc dây không gây chuyển vị lớn hơn 1 mm;
    
    \item Điểm móc dây phải là một vòng tròn có đường kính lỗ là (20 $\pm$ 1) mm và đường kính mặt cắt ngang là (15 $\pm$ 1) mm, hoặc một thanh truyền có đường kính mặt cắt ngang tương tự;
    
    \item Điểm móc dây phải có độ cao đủ để giữ được mẫu thử mô phỏng theo nửa thân người không bị rơi xuống sàn trong khi thử động.
\end{itemize}

\textbf{Quy định về thiết bị thử độ bền tĩnh:}

\begin{itemize}
    \item Bao gồm một khung thử, tời hoặc cơ cấu thủy lực, đồng hồ đo, và một thanh ngang phù hợp để tạo được tải lên mẫu thử mô phỏng theo nửa thân người.
\end{itemize}

\textbf{Quy định về dụng cụ đo lực:}

\begin{itemize}
    \item Phải có khả năng đo lực từ 1,2 kN đến 20 kN, với độ chính xác $\pm$ 2\%.
\end{itemize}

\textbf{Quy định về phương pháp thử nghiệm:}

\begin{enumerate}
    \item Lắp dây đai toàn thân lên hình nộm nửa người mô phỏng theo cách tương tự như mặc trên người, tuân theo hướng dẫn của nhà sản xuất. Điều chỉnh để đảm bảo dây đai ôm sát trên hình nộm.
    
    \item Đánh dấu vật liệu dây đai tại mỗi khóa điều chỉnh và khung khóa cài để các vết đánh dấu này được căn chỉnh.
    
    \item Tác dụng lực kéo 15 kN giữa phần tử gắn kết chống rơi và chốt treo dưới của hình nộm, với thời gian đạt đến lực này là (4 $\pm$ 1) phút. Duy trì lực trong 3 phút.
\end{enumerate}

\textbf{Một số quy định về đánh giá kết quả:}

Quan sát và ghi nhận nếu có:

\begin{itemize}
    \item Xé rách vật liệu vải làm đai
    \item Làm đứt một phần hoặc toàn bộ đường may ở bất kỳ điểm nào
    \item Gãy một phần hoặc toàn bộ của bất kỳ khóa điều chỉnh hoặc khóa nhanh nào
    \item Mở ra ngoài ý muốn của bất kỳ khóa nhanh nào
    \item Dịch chuyển lệch của các dây và các chi tiết liên kết khỏi vị trí ban đầu
    \item Lệch hàng của những dấu khóa
\end{itemize}
\section{Thiết bị hiện có trên thị trường}
Hiện tại, máy thử độ bền tĩnh cho dây đai theo TCVN 7802 (ISO 10333) chưa được sử dụng rộng rãi. Điều này chủ yếu do tính chất đặc thù của phương pháp thử nghiệm và nhu cầu thực tế từ các nhà sản xuất và cơ quan chứng nhận.

Thông thường, để đánh giá khả năng chịu tải và đảm bảo an toàn cho dây đai toàn thân, các nhà sản xuất ưu tiên sử dụng các hệ thống thử độ bền động quy mô lớn. Các hệ thống này mô phỏng các điều kiện thực tế khi một người sử dụng dây đai ở độ cao hoặc trong các tình huống có nguy cơ rơi tự do. Các bài thử nghiệm bao gồm mô phỏng rơi tự do với tải trọng hình nộm, đánh giá lực va đập tức thời, và độ giãn dài của dây đai dưới tải trọng.

Hơn nữa, do các yêu cầu nghiêm ngặt về an toàn và độ tin cậy, các hệ thống thử nghiệm động được thiết kế với cấu trúc vững chắc có khả năng kiểm soát chính xác các thông số như gia tốc, lực căng, và thời gian va đập. Các hệ thống này không chỉ đánh giá chất lượng sản phẩm một cách toàn diện mà còn hỗ trợ các nhà sản xuất cải thiện thiết kế và tối ưu hóa vật liệu để đáp ứng các tiêu chuẩn an toàn nghiêm ngặt.

\begin{figure}[H]
    \centering
    \includegraphics[width=0.5\textwidth]{pictures/chapter1/c1_p03.png}
    \caption{Ví dụ về hệ thống thử nghiệm tải trọng động cho dây bảo hộ}
    \label{fig:label}
\end{figure}

Đối với thử nghiệm độ bền tĩnh, máy nén kéo vạn năng (universal testing machines - UTMs) được sử dụng phổ biến trên thị trường để đánh giá chất lượng của dây đỡ cả người. Các máy này được thiết kế để thực hiện các thử nghiệm kéo, nén và uốn trên các loại vật liệu khác nhau, bao gồm dây đai an toàn trong lĩnh vực an toàn lao động.

Tuy nhiên, do các đặc thù của TCVN 7802 (ISO 10333), hầu hết các máy này chỉ có thể thử nghiệm các bộ phận riêng lẻ của dây đỡ cả người thay vì toàn bộ hệ thống trong điều kiện thực tế. Để phù hợp với thử nghiệm dây đỡ cả người, các máy UTM thường được trang bị các kẹp chuyên dụng để kiểm tra độ bền của các bộ phận chính như:

\begin{itemize}
    \item Dây đai chính: Xác định độ bền kéo và độ bền dưới tải trọng cao.
    \item Khóa mở nhanh: Đánh giá độ bền của khóa cài dưới lực tác động đột ngột hoặc kéo dài.
    \item Khóa điều chỉnh: Kiểm tra khả năng giữ cố định và không trượt trong quá trình sử dụng kéo dài.
    \item Dây đai kết nối: Đảm bảo độ bền và an toàn khi được kết nối với các điểm neo.
\end{itemize}

Mặc dù các máy kéo nén vạn năng có thể cung cấp kết quả chính xác cho các bộ phận riêng lẻ, chúng không thể mô phỏng hoàn toàn các điều kiện sử dụng thực tế của dây đai toàn thân. Điều này khiến chúng không thể thay thế hoàn toàn các hệ thống thử nghiệm chuyên dụng để đánh giá toàn bộ dây đai theo TCVN 7802.

\subsection{Máy thử độ bền kéo dây đai an toàn XJS108C}
Máy thử độ bền kéo dây đai an toàn XJ8108C, được thiết kế và sản xuất bởi Xiangjie Instrument, tuân thủ các tiêu chuẩn quốc gia GA 494-2004 cho thiết bị phòng cháy chữa cháy, GB6095-2009 cho dây đai an toàn, và GB/T6096-2009 cho phương pháp thử nghiệm dây đai an toàn.

Thiết bị này được thiết kế đặc biệt để thử nghiệm độ bền kéo tĩnh của dây đai chính trong hệ thống dây đai an toàn, đánh giá chất lượng và an toàn theo các tiêu chuẩn nghiêm ngặt. Với độ chính xác cao, hoạt động ổn định và độ tin cậy đặc biệt, XJ8108C được sử dụng rộng rãi trong các trung tâm chứng nhận sản phẩm về an toàn lao động, cơ quan phòng cháy chữa cháy, và các đơn vị giám sát an toàn khác nhau. Thiết bị này không chỉ hỗ trợ chứng nhận chất lượng sản phẩm mà còn giúp các nhà sản xuất tối ưu hóa thiết kế và nâng cao hiệu suất của dây đai an toàn trong các ứng dụng thực tế. \\
Sơ đồ nguyên lý của máy: 
\begin{figure}[H]
    \centering
    \includegraphics[width=0.5\textwidth]{pictures/chapter1/c1_p04_sodonguyenlyXJS108C.png}
    \caption{Sơ đồ nguyên lý máy XJS108C}
    \label{fig:label}
\end{figure}

\begin{table}[H]
    \centering
    \begin{tabular}{|p{5cm}|p{9cm}|}
        \hline
        \textbf{Mã sản phẩm} & XJ8108C \\
        \hline
        \textbf{Màn hình} & PC \\
        \hline
        \textbf{Tải trọng tối đa} & 100 kN \\
        \hline
        \textbf{Độ phân giải lực} & $\pm$1/50000 \\
        \hline
        \textbf{Độ chính xác lực} & $\leq$$\pm$0.2\%F.S \\
        \hline
        \textbf{Hành trình thử} & 1000mm \\
        \hline
        \textbf{Tốc độ thử} & 0.01-500mm/phút \\
        \hline
        \textbf{Chế độ dừng} & Dừng quá tải, dừng khẩn cấp, dừng khi mẫu hỏng, dừng giới hạn hành trình trên/dưới, tự động quay về \\
        \hline
        \textbf{Hệ thống truyền động chính} & Servo AC với hộp giảm tốc, truyền động đai răng và vít me chính xác \\
        \hline
        \textbf{Công suất} & 1500 W \\
        \hline
        \textbf{Loại nguồn} & AC 220V 50/60Hz \\
        \hline
        \textbf{Kích thước (R x S x C)} & 800x550x 1950mm \\
        \hline
        \textbf{Trọng lượng} & 500kg \\
        \hline
    \end{tabular}
    \caption{Thông số kỹ thuật máy XJ8108C}
    \label{tab:xj8108c_specs}
\end{table}

\subsection{Hệ thống thử độ bền kéo của công ty Fall Protection Laboratory (FPL)}
Fall Protection Laboratory (FPL), có trụ sở tại Mexico, là một công ty chuyên về thử nghiệm thiết bị bảo vệ rơi, bao gồm dây đai toàn thân. Với danh tiếng vững chắc trong ngành, FPL là một trong những cơ sở thử nghiệm hàng đầu tại quốc gia này, được công nhận theo tiêu chuẩn năng lực phòng thí nghiệm ISO 17025 và tiêu chuẩn EMA của Mexico, đảm bảo các phép đo và đánh giá đáp ứng tiêu chuẩn chất lượng quốc tế.
\begin{figure}[H]
    \centering
    \includegraphics[width=0.8\textwidth]{pictures/chapter1/c1_p07_FPL.png}
    \caption{Logo công ty Fall Protection Laboratory (FPL)}
    \label{fig:label}
\end{figure}
Mặc dù FPL không công khai thông số kỹ thuật chi tiết của các máy thử nghiệm độ bền tĩnh và động cho dây đai toàn thân, họ đã cung cấp nhiều hình ảnh về các hệ thống thử nghiệm của mình. Những hình ảnh này cho thấy các phương pháp thử nghiệm nghiêm ngặt và có hệ thống để đánh giá chất lượng sản phẩm. Trong thử nghiệm độ bền động, FPL sử dụng phương pháp mô phỏng rơi tự do để tái hiện các tình huống thực tế khi dây đai chịu tải trọng đột ngột. Hệ thống này đánh giá khả năng hấp thụ lực và độ bền của dây đai trong điều kiện khắc nghiệt, đảm bảo an toàn tối đa cho người sử dụng.

Đối với thử nghiệm độ bền tĩnh, FPL sử dụng các máy kéo chuyên dụng để đánh giá từng bộ phận riêng lẻ của hệ thống dây đai toàn thân, bao gồm dây đai chính, khóa kết nối và dây cáp liên kết. Phương pháp này cho phép đánh giá chính xác khả năng chịu tải của từng bộ phận, đảm bảo tuân thủ các tiêu chuẩn an toàn nghiêm ngặt. Bằng cách thử nghiệm từng bộ phận riêng lẻ, FPL có thể xác định các điểm yếu tiềm ẩn, giúp các nhà sản xuất cải thiện thiết kế và nâng cao chất lượng sản phẩm trước khi đưa vào sử dụng thực tế.

\begin{figure}[H]
    \centering
    \includegraphics[width=1\textwidth]{pictures/chapter1/c1_p05.png}
    \caption{Hệ thống thử độ bền kéo trong phòng thí nghiệm}
    \label{fig:label}
\end{figure}

\subsection{Máy thử dây đai an toàn điện tử KASON ETM504D}
\begin{figure}[H]
    \centering
    \includegraphics[width=0.7\textwidth]{pictures/chapter1/c1_p06_KASON.png}
    \caption{KASON ETM504}
    \label{fig:label}
\end{figure}   
Máy thử nghiệm dây đai an toàn điện tử điều khiển bằng máy tính dòng ETM (Không gian kép) được thiết kế và sản xuất theo các tiêu chuẩn ASTM, ISO, DIN, v.v. Đây là máy thử nghiệm chính xác điều khiển bằng máy tính, phù hợp với nhiều loại vật liệu để thử nghiệm kéo, nén, uốn, cắt, v.v. Máy có độ ổn định cao cũng như độ chính xác cao, được trang bị hệ thống PC và máy in để hiển thị đồ thị, kết quả thử nghiệm, in ấn và xử lý dữ liệu. Hoàn chỉnh với mô-đun cho kim loại, lò xo, dệt may, cao su, nhựa và các vật liệu khác. Máy được sử dụng rộng rãi trong nhiều lĩnh vực như nhà máy công nghiệp, doanh nghiệp khai khoáng và các trường đại học. 
\begin{table}[H]
    \centering
    \begin{tabular}{|p{6cm}|p{8cm}|}
        \hline
        \textbf{Mô hình} & ETM 504D \\
        \hline
        \textbf{Load cell} & \\
        \hline
        \textbf{Tiêu chuẩn hiệu chuẩn} & ISO 7500 \\
        \hline
        \textbf{Công suất tải (kN)} & 50 \\
        \hline
        \textbf{Cấp độ chính xác} & 0.5 \\
        \hline
        \textbf{Phạm vi lực thử nghiệm} & 0.2\%-100\%FS (toàn thang đo) \\
        \hline
        \textbf{Sai số chỉ báo lực thử} & Trong khoảng $\pm$0.5\% giá trị chỉ báo \\
        \hline
        \textbf{Độ phân giải lực thử} & $\pm$1/500000 lực thử tối đa, cấp độ không thay đổi và độ phân giải không đổi trong toàn bộ quá trình \\
        \hline
        \textbf{Phạm vi đo biến dạng} & Phạm vi đo biến dạng \\
        \hline
        \textbf{Sai số chỉ báo biến dạng} & Trong khoảng $\pm$0.5\% giá trị chỉ báo \\
        \hline
        \textbf{Độ phân giải biến dạng} & 1/300000 biến dạng tối đa \\
        \hline
        \textbf{Sai số chỉ báo dịch chuyển} & Trong khoảng $\pm$0.5\% giá trị chỉ báo \\
        \hline
        \textbf{Độ phân giải dịch chuyển} & 0.025$\mu$m \\
        \hline
        \textbf{Bộ điều khiển} & \\
        \hline
        \textbf{Tần số lấy mẫu} & Lên đến 1000 Hz \\
        \hline
        \textbf{Tần số điều khiển vòng kín} & Lên đến 1000Hz \\
        \hline
        \textbf{Độ phân giải} & 20 bit \\
        \hline
        \textbf{Phần mềm} & KASON TEST (Hỗ trợ tiếng Anh, tiếng Nga, tiếng Thổ Nhĩ Kỳ, v.v.) \\
        \hline
        \textbf{Khung chính} & \\
        \hline
        \textbf{Phạm vi tốc độ (mm/phút)} & 0.01-500 \\
        \hline
        \textbf{Độ chính xác tốc độ thử} & $\pm$0.5\% \\
        \hline
        \textbf{Không gian thử kéo (mm)} & 2500 \\
        \hline
        \textbf{Chiều rộng thử (mm)} & 700 \\
        \hline
        \textbf{Đĩa nén (mm)} & $\Phi$90mm \\
        \hline
        \textbf{Nguồn điện} & Đơn pha AC220V$\pm$10\%, 50Hz/60Hz \\
        \hline
        \textbf{Nhiệt độ hoạt động} & 0 đến +38$^{\circ}$ C \\
        \hline
    \end{tabular}
    \caption{Thông số kỹ thuật máy ETM 504D}
    \label{tab:etm504d_specs}
\end{table}

\cleardoublepage
\section{Mục tiêu, nhiệm vụ, phạm vi và phân công thiết kế}
\subsection{Mục tiêu}
Thiết kế và chế tạo thiết bị thử nghiệm độ bền tĩnh cho dây đỡ cả người với các yêu cầu sau:
\begin{itemize}
    \item Thử nghiệm độ bền tĩnh của dây đỡ cả người.
    \item Đảm bảo kết quả thử nghiệm và quy trình tuân thủ TCVN 7802-1:2007.
\end{itemize}
    
\subsection{Nhiệm vụ}
\begin{itemize}
    \item Phân tích và đề xuất sơ đồ nguyên lý, sơ đồ động học, và các phương án thiết kế cơ khí, điện, và điều khiển cho thiết bị.
    \item Tính toán và thiết kế các bộ phận cơ khí, bao gồm hệ thống truyền động và gá đỡ cảm biến, để đáp ứng các điều kiện thử nghiệm độ bền kéo quy định trong TCVN 7802-1:2007.
    \item Tính toán và thiết kế hệ thống điện của máy.
    \item Thiết kế hệ thống điều khiển thiết bị.
    \item Tạo mô hình 3D, bản vẽ lắp ráp cơ khí, bản vẽ điện, sơ đồ điều khiển, và mô hình thử nghiệm.
    \item Thử nghiệm và đánh giá thiết bị.
\end{itemize}
    
\subsection{Phạm vi đề tài}
\begin{itemize}
    \item Thiết kế thiết bị thử nghiệm độ bền tĩnh cho dây đai toàn thân sử dụng trong an toàn lao động.
    \item Đảm bảo thiết bị tuân thủ quy định thử nghiệm TCVN 7802-1:2007.
\end{itemize}
    
\subsection{Phân công thiết kế}
Thiết kế máy thử nghiệm độ bền tĩnh cho dây đai toàn thân theo TCVN 7802-1:2007 cho "Dây đai toàn thân" với các thông số kỹ thuật sau:
\begin{itemize}
    \item Lực tối đa tác dụng lên dây đai an toàn: 15 kN
    \item Phạm vi hoạt động của thiết bị đo lực: 1,2 kN đến 20 kN
    \item Độ chính xác của thiết bị đo lực: $\pm$ 2\%;
    \item Có khả năng tăng lực lên 15 kN trong vòng 240 giây và duy trì lực trong 180 giây
\end{itemize}
    